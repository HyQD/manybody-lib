\documentclass{article}
\usepackage{graphicx} % Required for inserting images
\usepackage{amsmath}
\usepackage{braket}
\usepackage[hidelinks]{hyperref}
\usepackage[natbib=true,style=phys,sorting=none]{biblatex}
\addbibresource{refs.bib}

\title{Closed-shell parametrization}

\date{\today}

\begin{document}

\maketitle

\section{Notation}

\subsection{Excitation operators}
Let $\hat{a}^\dagger_{p,\sigma} (\hat{a}_{p,\sigma})$ be the creation (annihilation) operator associated with a spin-orbital
\begin{equation}
    \psi_p(\mathbf{x}) \equiv \phi_p(\mathbf{r}) \otimes \sigma(m_s),
\end{equation}
where $\phi_p(\mathbf{r})$ is a spatial orbitals, and $\sigma(m_s)$ a spin eigenfunction. The spin functions are orthonormal in the sense 
\begin{equation}
    \int \sigma^*(m_s) \tau(m_s) dm_s = \delta_{\sigma, \tau}.
\end{equation}

Following \citeauthor{helgaker2013molecular}~\cite{helgaker2013molecular}, the singlet excitation operators are defined as
\begin{align}
    \hat{E}_{pq} &\equiv \hat{a}_{p,\alpha}^\dagger \hat{a}_{q,\alpha} + \hat{a}_{p, \beta}^\dagger \hat{a}_{q, \beta},
\end{align}
and satisfy the commutation relations
\begin{align}
    [\hat{E}_{pq}, \hat{E}_{rs}] &= \hat{E}_{ps} \delta_{rq} - \hat{E}_{rq} \delta_{ps} \label{comm_Epq_Ers}.
\end{align}
Moreover, we introduce the two-electron excitation operator
\begin{equation}
    \hat{e}_{pqrs} \equiv \hat{E}_{pr} \hat{E}_{qs} - \delta_{rq} \hat{E}_{ps} = \sum_{\sigma \tau} \hat{a}_{p,\sigma}^\dagger \hat{a}_{q, \tau}^\dagger \hat{a}_{s, \tau} \hat{a}_{r, \sigma}, 
\end{equation}
which satisfy the following commutation relation with $\hat{E}_{t,u}$
\begin{equation}
    [\hat{e}_{pqrs}, \hat{E}_{tu}] = \hat{e}_{pqru} \delta_{ts} - \hat{e}_{ptrs}\delta_{qu} + \hat{e}_{pqus}\delta_{tr} - \hat{e}_{tqrs} \delta_{pu} \label{comm_epqrs_Etu}.
\end{equation}

\subsection{The Hamiltonian}
The second quantized representation of a nonrelativistic spin-free Hamiltonian containing at most two-particle interactions can now be written in terms of singlet and two-electron excitation operators as
\begin{equation}
    \hat{H} = \sum_{pq}h^p_q \hat{E}_{pq} + \frac{1}{2} \sum_{pqrs} u^{pq}_{rs} \hat{e}_{pqrs} \label{eq:spin_free_Hamiltonian},
\end{equation}
where the one- and two-body matrix elements $h^p_q$ and $u^{pq}_{rs}$ (physicist notation) are defined as 
\begin{align}
    h^p_q &\equiv \int \phi_p^*(\mathbf{r}) \hat{h} \phi_q(\mathbf{r}) d\mathbf{r}, \\
    u^{pq}_{rs} &\equiv \iint \phi_p^*(\mathbf{r}_1) \phi_q^*(\mathbf{r}_2) \hat{w}(\mathbf{r}_1, \mathbf{r}_2) \phi_r(\mathbf{r}_1) \phi_s(\mathbf{r}_2) d\mathbf{r}_1 d\mathbf{r}_2,  
\end{align}
where $\hat{h}$ is a one-body operator and $\hat{w}(\mathbf{r}_1, \mathbf{r}_2)$ is a two-particle interaction/operator (typically the electron-electron Coulomb interaction).

The expectation value of the (spin-free) Hamiltonian for a (closed-shell) reference determinant (usually the Hartree-Fock determinant) is given by
\begin{align}
    E_{\text{ref}} \equiv \braket{\Phi_0|\hat{H}|\Phi_0} = 2\sum_{i} h^i_i + 2\sum_{ij} u^{ij}_{ij}-\sum_{ij} u^{ij}_{ji},
\end{align}
where $i,j$ run over all (doubly) occupied orbitals $\{ \phi_i(\mathbf{r}) \}_{i=1}^{N_{\text{docc}}}$. More generally, the expectation value of the Hamiltonian (or some other one- plus two-body operator) where $\bra{\tilde{\Psi}}, \ket{\Psi}$ are (possibly independent) bra and ket states can be written as
\begin{equation}
    \braket{\tilde{\Psi}|\hat{H}|\Psi} = \sum_{pq} h^p_q \gamma^q_p + \frac{1}{2} \sum_{pqrs} u^{pq}_{rs} \Gamma_{pq}^{rs},
\end{equation}
where we have defined the one- and two-body reduced density matrices
\begin{align}
    \gamma^q_p &\equiv \braket{\tilde{\Psi}|\hat{E}_{pq}|\Psi}, \\
    \Gamma^{rs}_{pq} &\equiv \braket{\tilde{\Psi}|\hat{e}_{pqrs}|\Psi}.
\end{align}
The expressions for the reduced density matrices depend on the particular wavefunction parametrization used, i.e., Hartree-Fock, configuration interaction, coupled cluster, and approximate coupled cluster methods such as CC2 or OMP2.

\subsection{Singlet configuration state functions}
Single and double excited (ket/bra) (singlet) configuration state functions (CSFs) are defined as 
\begin{align}
   \ket{\Phi^a_i} &\equiv \hat{X}^a_i\ket{\Phi_0}, \ \ \ket{\Phi^{ab}_{ij}} \equiv \hat{X}^{ab}_{ij}\ket{\Phi_0}, \\
   \bra{\Phi^a_i} &\equiv \bra{\Phi_0}E_{ia}, \ \ \bra{\Phi^{ab}_{ij}} \equiv \bra{\Phi_0}E_{jb}E_{ia},
\end{align}
where we have defined the single and double excitation (de-excitation) operators
\begin{align}
    \hat{X}^a_i &\equiv \hat{E}_{ai}, \ \ \hat{X}^{ab}_{ij} \equiv \hat{E}_{ai} \hat{E}_{bj}, \\
    (\hat{X}^a_i)^\dagger &= \hat{E}_{ia}, \ \  (\hat{X}^{ab}_{ij})^\dagger = \hat{E}_{jb} \hat{E}_{ia}.
\end{align}

The CSFs defined this way are not orthogonal, and the overlaps are given by 
\begin{align}
    \braket{\Phi^a_i|\Phi^b_j} &= 2 \delta_{a b} \delta_{i j}, \\
    \braket{\Phi^{ab}_{ij}|\Phi^{cd}_{kl}} &= 2 \hat{P}^{ab}_{ij}\left( 2\delta_{a c} \delta_{b d} \delta_{i k} \delta_{j l}-\delta_{a c} \delta_{b d} \delta_{i l} \delta_{j k} \right),
\end{align}
where we have defined the permutation operator 
\begin{equation}
    \hat{P}^{pq}_{rs} A^{pq}_{rs} = A^{pq}_{rs} + A^{qp}_{sr}. 
\end{equation}

However, following Refs.~\cite{helgaker2013molecular, pulay1984efficient} a biorthogonal basis can be defined (for singles and doubles), such that,
\begin{align}
    \braket{\tilde{\Phi}^a_i|\Phi^c_k} &= \delta_{ac} \delta_{ik}, \\
    \braket{\tilde{\Phi}^{ab}_{ij}|\Phi^{cd}_{kl}} &= P^{ab}_{ij} \delta_{ac}\delta_{bd}\delta_{ik}\delta_{jl},
\end{align}
by choosing 
\begin{align}
    \bra{\tilde{\Phi}^a_i} &= \frac{1}{2} \bra{\Phi^a_i} = \bra{\Phi_0} \hat{Y}^i_a, \\
    \bra{\tilde{\Phi}^{ab}_{ij}} &= \frac{1}{3} \bra{\Phi^{ab}_{ij}} + \frac{1}{6} \bra{\Phi^{ab}_{ji}} = \bra{\Phi_0}\hat{Y}^{ij}_{ab}, 
\end{align}
where we have defined the single and double de-exciation operators
\begin{align}
    \hat{Y}^i_a &= \frac{1}{2} \hat{E}_{ia}, \\
    \hat{Y}^{ij}_{ab} &= \frac{1}{3} \hat{E}_{jb} \hat{E}_{ia} + \frac{1}{6} \hat{E}_{ib} \hat{E}_{ja} 
\end{align}

\section{Configuration interaction}
\subsection{CISD}
We use the following parametrization of the CISD ket and bra wavefunction
\begin{align}
    \ket{\Psi} &= C_0 \ket{\Phi_0} + \sum_{ai} C^a_i \ket{\Phi^a_i} + \frac{1}{2} \sum_{abij} C^{ab}_{ij} \ket{\Phi^{ab}_{ij}}, \\ 
    \bra{\tilde{\Psi}} &= \bra{\tilde{\Phi}_0} \tilde{C}_0 + \sum_{ia} \tilde{\Phi}^a_i \tilde{C}^i_a + \frac{1}{2} \sum_{ijab} \bra{\tilde{\Phi}^{ab}_{ij}}\tilde{C}^{ij}_{ab}, 
\end{align}
where the left expansion coefficients $\tilde{C}$ are related to the right coefficients $C$ by
\begin{align}
    \tilde{C}_0 &= C_0^*, \\
    \tilde{C}^i_a &= 2 C^{a *}_i,
    \tilde{C}^{ij}_{ab} = \left(4C^{ab}_{ij}-2C^{ab}_{ji} \right)^*.
\end{align}
The norm/overlap of the bra and ket states is given by
\begin{equation}
    \braket{\tilde{\Psi}|\Psi} = \tilde{C}_0 C_0 + \sum_{ai} \tilde{C}^i_a C^a_i + \frac{1}{2} \sum_{abij} \tilde{C}^{ij}_{ab} C^{ab}_{ij},
\end{equation}
Right and left sigma-vectors are defined as
\begin{align}
\sigma_\mu &\equiv \sum_\nu \braket{\tilde{\Phi}_\mu|\hat{H}|\Phi_\nu} C_\nu, \\ 
\tilde{\sigma}_\nu &\equiv \sum_\mu \tilde{C}_\mu \braket{\tilde{\Phi}_\mu|\hat{H}|\Phi_\nu}.
\end{align}

Working equations/algebraic expressions for the (closed-shell) sigma-vectors and the one- and two-body density matrices can be generated symbolically with the \texttt{drudge/gristmill} package~\cite{DrudgeCAS, Zhao2018Drudge}.

The CIS and CID approximation are obtained by setting the doubles or singles coefficients to zero, respectively.

\section{Coupled cluster}
\subsection{CCSD}
The right and left CCSD wavefunctions are parametrized as 
\begin{align}
    \ket{\Psi} = e^{\hat{T}}\ket{\Phi_0}, \ \ \bra{\tilde{\Psi}} = \bra{\Phi_0}\left(1+\hat{\Lambda} \right)e^{-\hat{T}},
\end{align}
where we have defined the cluster excitation (de-excitation) operators
\begin{align}
    \hat{T} &= \hat{T}_1 + \hat{T}_2, \ \
    \hat{\Lambda} = \hat{\Lambda}_1 + \hat{\Lambda}_2 \\
    \hat{T}_1 &= \sum_{ai} \tau^a_i \hat{X}^a_i, \ \  \hat{T}_2 = \frac{1}{2} \sum_{abij} \tau^{ab}_{ij} \hat{X}^{ab}_{ij}, \\
    \hat{\Lambda}_1 &= \sum_{ia} \lambda^i_a \hat{Y}^i_a, \ \ \hat{\Lambda}_2 = \frac{1}{2} \sum_{ijab} \lambda^{ij}_{ab} \hat{Y}^{ij}_{ab},
\end{align}
and the CCSD energy functional is given by
\begin{equation}
    \mathcal{H} = \braket{\tilde{\Psi}|\hat{H}|\Psi}.
\end{equation}

The CCSD-amplitudes $(\lambda, \tau)$ (for the ground state) are determined by solving the (non-linear) amplitude equations
\begin{align}
    \frac{\partial \mathcal{H}}{\partial \lambda_\mu} &= \braket{\Phi_0|\hat{Y}^\mu e^{-\hat{T}} \hat{H}e^{\hat{T}}|\Phi_0} = 0, \\
    \frac{\partial \mathcal{H}}{\partial \tau^\mu} &= \braket{\Phi_0|\left(1+\hat{\Lambda}\right)e^{-\hat{T}}[\hat{H}, \hat{X}_\mu]e^{\hat{T}}|\Phi_0} = 0,
\end{align}
where $\mu$ denotes the excitation level (singles or doubles).

\begin{align}
    \gamma_p^q &\equiv \braket{\tilde{\Psi}|\hat{E}_{pq}|\Psi}, \\
    \Gamma_{pq}^{rs} &\equiv \braket{\tilde{\Psi}|\hat{e}_{pqrs}|\Psi}
\end{align}

Excited states can be determined with the EOM-CCSD approach, where the left and right excited states are parametrized as 
\begin{align}
    \ket{\Psi_k} &= \hat{R}_k e^{\hat{T}} \ket{\Phi_0}, \\
    \bra{\tilde{\Psi}_k} &= \bra{\tilde{\Phi}_0}\hat{L}_ke^{-\hat{T}},
\end{align}
where 
\begin{equation}
    \hat{R}_k = \sum_{\mu=0} r_k^\mu \hat{X}_\mu, \ \ \hat{L}_k = \sum_{\mu=0} l_{k,\mu} \hat{Y}^\mu.
\end{equation}
The EOM coefficients $l, r$ are determined by solving the left and right eigenvalue problems
\begin{align}
    \sigma_\mu &=  E_k r^\mu_k, \\
    \tilde{\sigma}_\nu &= E_k l_{\nu,k} 
\end{align}
where we have defined the EOM sigma-vectors
\begin{align}
     \sum_{\nu} \Bar{H}_{\mu,\nu} r^\nu_k &\equiv \sigma_\mu \label{sigma-EOM},\\
   \sum_{\mu} l_{k,\mu}\Bar{H}_{\mu,\nu} &\equiv \tilde{\sigma}_\nu \label{tilde-sigma-EOM},
\end{align}
and the matrix elements of the similarity transformed Hamiltonian $\Bar{H} \equiv e^{-\hat{T}}\hat{H}e^{\hat{T}}$
\begin{equation}
    \Bar{H}_{\mu, \nu} \equiv \braket{\Phi_\mu|e^{-\hat{T}}\hat{H}e^{\hat{T}}|\Phi_\nu}.
\end{equation}
Moreover, the right and left eigenvectors can be normalized such that they form a bi-orthonormal set, i.e., 
\begin{equation}
    \braket{\tilde{\Psi}_k|\Psi_l} = \delta_{k,l}.
\end{equation}
Working equations for the (closed-shell) amplitude equations and the sigma vectors can be generated with the \texttt{drudge/gristmill} package.

\begin{appendix}

\end{appendix}
\printbibliography
\end{document}
